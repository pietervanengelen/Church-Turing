\documentclass{beamer}
%\usetheme{Boadilla}
\usepackage{ru}
\usepackage{amsmath}
\usepackage{chronosys}
\usepackage{tikz}

\usetikzlibrary{arrows, calc, decorations.markings, positioning}

\input definitions_slides.tex

\title{Church-Turing These}
\subtitle{Een nieuw paradijs}
\author{Pieter van Engelen}
\institute{Radboud Universiteit Nijmegen}
\date{03-06-2022}

\begin{document}

\begin{frame}
    \titlepage
\end{frame}

\begin{frame}
    \tableofcontents
\end{frame}

\section{De tijd}
\begin{frame}
    \begin{timeline}{1925}{1950}{2cm}{2.5cm}{7cm}{10cm}
        \entry{1928}{\small Formulering \emph{Entscheidungsproblem} in \emph{Grundzüge der theoretischen Logik}}
        \entry{1931}{\small Onvolledigheidsstellingen van Gödel}
        \entry{1932}{\small Introductie van de $\lambda$-calculus (A. Church)}
        \entry{1936}{\small Negatieve oplossing \emph{Entscheidungsproblem}}
        \entry{1937}{\small Introductie \emph{turing machine, halting problem} en negatieve oplossing \emph{Entscheidungsproblem}}
        \entry{1939}{\small \emph{Systems of logics based on ordinals} (Turing)}
        \entry{1944}{\small \emph{Introductie Turing-degree} (E. Post)}
        \entry{1945}{\small \emph{First Draft of a Report on the EDVAC} (von Neumann)}
    \end{timeline}    
\end{frame}

\begin{frame}
    \frametitle{De These}
    \begin{center}
        {\Large
            Every \emph{effectively calculable} function is \emph{computable}
        }
        \\
        \bigskip
        \bigskip
        Church (1936), Turing (1937)
    \end{center}
\end{frame}

\subsection{De protagonisten}
\begin{frame}
    \frametitle{De protagonisten}
    \begin{columns}
        \column{0.5\textwidth}
        \includegraphics[width=\textwidth]{Church.jpeg}
        \column{0.5\textwidth}
        {\Large Alonzo Church} (1903 - 1995) 

        \emph{Princeton University, USA}
        \begin{itemize}
            \item Logicus, wiskundige
            \item Van 1936 tot 1979 redacteur van \emph{Journal of Symbolic Logic}
            \item 'Bedenker' van de $\lambda$-calculus
            \item Eerste-orde predicaat-logica is onbeslisbaar
            \item Peano-arithmetiek is onbeslisbaar
        \end{itemize}    
    \end{columns}
\end{frame}

\begin{frame}
    \frametitle{De protagonisten}
    \begin{columns}
        \column{0.5\textwidth}
        \includegraphics[width=\textwidth]{Turing.jpg}
        \column{0.5\textwidth}
        {\Large Alan Turing} (1912 - 1954) 

        \emph{Cambridge \& Manchester}
        \begin{itemize}
            \item Grondlegger van
            \begin{itemize}
                \item Informatica
                \item Artificiële intelligentie
                \item Morphogenetica
            \end{itemize}
            \item Legendarisch codebreaker
            \item Marathonloper
        \end{itemize}    
    \end{columns}
\end{frame}

\begin{frame}
    \frametitle{De protagonisten}
    \begin{tabular*}{\textwidth}{c c}
%%    \begin{columns}
%        \column{0.5\textwidth}
%        \includegraphics[width=0.8\textwidth]{Kleene.jpeg}
%        {\large Stephen Kleene} (1909-1994)
%        \column{0.5\textwidth}
%        \includegraphics[width=0.8\textwidth]{example-image-a}
%        {\large ???} (1897 - 1954)
%    \end{columns}
        \includegraphics[width=0.4\textwidth]{Kleene.jpeg} & \includegraphics[width=0.4\textwidth]{example-image-duck} \\
        {\large Stephen Kleene} (1909-1994) & {\large ???} (1897 - 1954)\\
    \end{tabular*}
\end{frame}

\section{De situatie}
\subsection{Entscheidungsproblem}
\begin{frame}
    \frametitle{Das Entscheidungsproblem}
\end{frame}

\subsection{Berekenbaarheidsmodellen}
\begin{frame}
    \frametitle{De $\lambda$-calculus}
\end{frame}

\begin{frame}
    \frametitle{Recursietheorie}
\end{frame}

\begin{frame}
    \frametitle{Turing machines}
\end{frame}

\begin{frame}
    \frametitle{De equivalentie}

    $$\lambda-\text{definieerbaar} \stackrel{\scriptscriptstyle \text{(Turing 1937)}}{\Longrightarrow} \text{Turing berekenbaar}$$

    $$\text{Turing berekenbaar} \stackrel{\scriptscriptstyle \text{(Turing 1937)}}{\Longrightarrow} \mu-\text{recursief} $$

    $$\mu-\text{recursief} \stackrel{\scriptscriptstyle \text{(Kleene 1936)}}{\Longrightarrow} \lambda-\text{definieerbaar} $$
\end{frame}

\subsection{De kracht van berekenbaarheid}
\begin{frame}
    \frametitle{Halting Problem}
\end{frame}

\begin{frame}
    \frametitle{Universaliteits principe}
\end{frame}

\section{De these}
\begin{frame}
    \frametitle{De These}
    \begin{center}
        {\Large
            Every \emph{effectively calculable} function is \emph{computable}
        }
        \\
        \bigskip
        \bigskip
        Church (1936), Turing (1937)
    \end{center}
\end{frame}


\section{Huidige stand van zaken}
\subsection{Hypercomputation}
\begin{frame}
    \frametitle{Hypercomputation}
\end{frame}
\subsection{Quantum computing}
\begin{frame}
    \frametitle{Quantum computing}
\end{frame}

\end{document}