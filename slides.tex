\documentclass{beamer}
%\usetheme{Boadilla}
\usepackage{ru}
\usepackage{amsmath}
%\usepackage{chronosys}
\usepackage{tikz}
\usepackage{epigraph}
\usepackage{xcolor}


\usetikzlibrary{arrows, calc, decorations.markings, positioning}

\usefonttheme[onlymath]{serif}

\input definitions_slides.tex

\title{Church-Turing These}
\subtitle{Een nieuw paradijs}
\author{Pieter van Engelen}
\institute{Radboud Universiteit Nijmegen}
\date{03-06-2022}

\begin{document}

\begin{frame}
    \titlepage
\end{frame}

\begin{frame}
    \tableofcontents
\end{frame}

\section{De tijd}
\begin{frame}
    \begin{timeline}{1925}{1950}{2cm}{2.5cm}{7cm}{10cm}
        \entry{1928}{\small Formulering \emph{Entscheidungsproblem} in \emph{Grundzüge der theoretischen Logik}}
        \entry{1931}{\small Onvolledigheidsstellingen van Gödel}
        \entry{1932}{\small Introductie van de $\lambda$-calculus (A. Church)}
        \entry{1936}{\small Negatieve oplossing \emph{Entscheidungsproblem}}
        \entry{1937}{\small Introductie \emph{turing machine, halting problem} en negatieve oplossing \emph{Entscheidungsproblem}}
        \entry{1939}{\small \emph{Systems of logics based on ordinals} (Turing)}
        \entry{1944}{\small \emph{Introductie Turing-degree} (E. Post)}
        \entry{1945}{\small \emph{First Draft of a Report on the EDVAC} (von Neumann)}
    \end{timeline}    
\end{frame}

\begin{frame}
    \frametitle{De These}
    \begin{center}
        {\Large
            Every \emph{effectively calculable} function is \emph{computable}
        }
        \\
        \bigskip
        \bigskip
        Church (1936), Turing (1937)
    \end{center}
\end{frame}

\subsection{De protagonisten}
\begin{frame}
    \frametitle{De protagonisten}
    \begin{columns}
        \column{0.5\textwidth}
        \includegraphics[width=\textwidth]{Church.jpeg}
        \column{0.5\textwidth}
        {\Large Alonzo Church} (1903 - 1995) 

        \emph{Princeton University, USA}
        \begin{itemize}
            \item Logicus, wiskundige
            \item Van 1936 tot 1979 redacteur van \emph{Journal of Symbolic Logic}
            \item 'Bedenker' van de $\lambda$-calculus
            \item Eerste-orde predicaat-logica is onbeslisbaar
            \item Peano-arithmetiek is onbeslisbaar
        \end{itemize}    
    \end{columns}
\end{frame}

\begin{frame}
    \frametitle{De protagonisten}
    \begin{columns}
        \column{0.5\textwidth}
        \includegraphics[width=\textwidth]{Turing.jpg}
        \column{0.5\textwidth}
        {\Large Alan Turing} (1912 - 1954) 

        \emph{Cambridge \& Manchester}
        \begin{itemize}
            \item Grondlegger van
            \begin{itemize}
                \item Informatica
                \item Artificiële intelligentie
                \item Morphogenetica
            \end{itemize}
            \item Legendarisch codebreaker
            \item Marathonloper
        \end{itemize}    
    \end{columns}
\end{frame}

\begin{frame}
    \frametitle{De protagonisten}
    \begin{tabular*}{\textwidth}{c c}
        \includegraphics[width=0.4\textwidth]{Kleene.jpeg} & \includegraphics[width=0.4\textwidth]{example-image-duck} \\
        {\large Stephen Kleene} (1909-1994) & {\large ???} (1897 - 1954)\\
    \end{tabular*}
\end{frame}

\section{De situatie}
\subsection{Entscheidungsproblem}
\begin{frame}
    \frametitle{Das Entscheidungsproblem}
    {\large \textbf{Das Entscheidungsproblem}}

    \begin{center}
        Vind een algoritme waarmee \\ 
        de waarheid van een uitspraak in de eerste orde predikaatlogica \\
        vast te stellen is.
        \\
        \bigskip
        {\small \emph{(D. Hilbert \& W. Ackermann, 1928, Grundzüge der theoretischen Logik)}}
    \end{center}
\end{frame}

\begin{frame}
    \frametitle{Entscheidungsproblem}
    \textbf{Eerste orde predikaatlogica} \\
    (extreem kort door de bocht) 
    \bigskip
    
    Logica met
    \begin{itemize}
        \item variabelen
        \item de gebruikelijke operatoren $\wedge, \vee, \rightarrow, \neg, \ldots$
        \item predikaten $P(x)$
        \item universele en existentiële kwantificatie $\forall, \exists$
    \end{itemize}

    Voorbeelden:
    $$\forall_{n \in \mathbb{N}}\exists_{m \in \mathbb{N}} [m>n]$$
    $$\forall_{p, q \in \mathbb{Q}} \exists_{r \in \mathbb{Q}}  [ p < r < q]$$
    $$\exists_x [P(x)\wedge \forall_y \forall_{y'}[P(y) \wedge P(y') \rightarrow y = y']]$$
\end{frame}

\begin{frame}
    \frametitle{Entscheidungsproblem}
    \textbf{Eerste orde predikaatlogica} \\

    \emph{Afspraak:}

    We hebben het alleen over predikaten en kwantificatie over de natuurlijke getallen $\mathbb{N}$

    \vspace{1cm}

    \emph{Gezocht: }
    
    \textbf{Algoritme} wat gegeven een uitspraak roept of die uitspraak \texttt{WAAR} of \texttt{ONWAAR} is.

    \vspace{1cm}
    \emph{Probleem: }
    
    Wat is een algoritme?
\end{frame}

\subsection{Berekenbaarheidsmodellen}
\begin{frame} 
    \frametitle{Wat is een algoritme}
    {\Large Wat is een algoritme??}
    \vspace{1cm}
    \begin{itemize}
        \item<1-> Grootste-gemene-deler van Euclides
        \item<2-> Zeef van Eratosthenes
        \item<3-> Gauss-eliminatie
    \end{itemize}
\end{frame}
\begin{frame} 
    \frametitle{Wat is een algoritme}

    \textbf{Probleem:} Nog geen \emph{formele} definitie van een \emph{algoritme}.
    \\
    \vspace{1cm}
    \begin{onlyenv}<2->
        Terug naar 1936\only<3->{-ish}.
        \begin{itemize}
            \item<4-> Turing machines
            \item<5-> Recursietheorie
            \item<6-> $\lambda$-calculus
        \end{itemize}        
    \end{onlyenv}
\end{frame}

\begin{frame}
    \frametitle{De $\lambda$-calculus (\emph{Church 1932})}
    \textbf{De programma's}
    \begin{align*}
        x,y,\ldots \in \Lambda & \text{  (Variabelen)}\\
        M,N\in \Lambda \Rightarrow MN \in \Lambda & \text{  (Applicatie)} \\
        x, M\in \Lambda \Rightarrow (\lambda x.M) \in \Lambda & \text{  (Abstractie)}
    \end{align*}
    \begin{itemize}
        \item $\lambda x.x$
        \item $\lambda xy.x$
        \item $\lambda pqr.pr(qr)$
        \item $(\lambda x.xx)A$
        \item $\lambda x.y$
        \item $\lambda fx.f(f(f(x))) \equiv \ulcorner 3 \urcorner$
    \end{itemize}    
\end{frame}

\begin{frame}
    \frametitle{De $\lambda$-calculus (\emph{Church 1932})}

    \textbf{Actie}

    $$(\lambda x.M)N \longrightarrow_\beta M [x:=N]$$

    \begin{onlyenv}<2->
            
        \textbf{Voorbeeld}
        
        \begin{align*}
            (\lambda xyz.zxy)(\mathcolor{red}{\lambda x.xx})(\mathcolor{blue}{\lambda x.x})(\lambda xy.x) & \rightarrow_\beta \\
            \only<3->{(\lambda yz.z(\mathcolor{red}{\lambda x.xx})y)(\mathcolor{blue}{\lambda x.x})(\lambda xy.x) & \rightarrow_\beta \\}
            \only<4->{(\lambda z.z(\mathcolor{red}{\lambda x.xx}))\mathcolor{blue}{\lambda x.x}(\lambda xy.x) & \rightarrow_\beta \\}
            \only<5->{(\lambda xy.x)(\mathcolor{red}{\lambda x.xx}))\mathcolor{blue}{\lambda x.x} & \twoheadrightarrow_\beta  \mathcolor{red}{\lambda x.xx}\\}
        \end{align*}
    \end{onlyenv}
\end{frame}

\begin{frame}
    \frametitle{Recursietheorie (\emph{Kleene 1935})}
    \textbf{Initiële functies}
    \begin{align*}
        \mathcal{O}(x) & = 0 & \text{Nul}\\
        \mathcal{S}(x) & = x + 1 & \text{Successor}\\
        \mathcal{P}^n_i(x_1, \ldots, x_n) & = x_i & \text{Projectie} \\
        f(\vec{x}) & = h(g_1(\vec{x}), \ldots, g_m(\vec{x})) & \text{Functie compositie}
    \end{align*}
    \textbf{Primitieve recursie}
    \begin{align*}
        f(\vec{x},0)   & = g(\vec{x}) & \text{0-geval} \\
        f(\vec{x},n+1) & = h(\vec{x}, y, f(\vec{x},y)) & \text{Recursieve geval}
    \end{align*}
    \textbf{$\mu$-recursie}
    \begin{align*}
        f(\vec{x}) & = \mu y[g(\vec{x},y)=0] \\
           & \text{"De kleinste $y$ zodat $g(\vec{x},y)=0$"}
    \end{align*}
\end{frame}

\begin{frame}
    \frametitle{Recursietheorie (\emph{Kleene 1935})}
    \textbf{Voorbeelden}
    \begin{columns}
        \column{0.5\textwidth}
            \begin{align*}
                \mathcal{P} (0) & = 0 \\
                \mathcal{P} (n + 1) & = n
            \end{align*}
        \column{0.5\textwidth}
            \begin{align*}
                \min (x,0) & = x \\
                \min (x, y+1) & = \mathcal{P}(\min (x,y))
            \end{align*}
        \end{columns}
        \vspace{1cm}
        $$f(n) = \mu y [2y= n \vee 2y+1 = n]$$
\end{frame}

\begin{frame}
    \frametitle{Turing machines (\emph{Turing 1936})}
    \begin{columns}
        \column{0.5\textwidth}
        \column{0.5\textwidth}
            \includegraphics[width=\textwidth]{tm.png}
    \end{columns}
\end{frame}

\begin{frame}
    \frametitle{De equivalentie}

    $$\lambda-\text{definieerbaar} \stackrel{\scriptscriptstyle \text{(Turing 1937)}}{\Longrightarrow} \text{Turing berekenbaar}$$

    $$\text{Turing berekenbaar} \stackrel{\scriptscriptstyle \text{(Turing 1937)}}{\Longrightarrow} \mu-\text{recursief} $$

    $$\mu-\text{recursief} \stackrel{\scriptscriptstyle \text{(Kleene 1936)}}{\Longrightarrow} \lambda-\text{definieerbaar} $$
\end{frame}

\begin{frame}
    \frametitle{De equivalentie}

    De uitspraken:
    \begin{itemize}
        \item Een functie $f:\mathbb{N} \rightarrow \mathbb{N}$ is berekenbaar
        \item Er bestaat een $\lambda$-term $F$ zdd $f(n) = m \Leftrightarrow F \ulcorner n\urcorner = \ulcorner m\urcorner $
        \item Er bestaat een $\mu$-recursieve functie $\phi$ zdd $f(n) = m \Leftrightarrow \phi(n) = m$
        \item Er bestaat een T.M. zdd $f(n) = m \Leftrightarrow \text{T.M.}_f \text{ geeft bij invoer } \ulcorner n\urcorner \text{ uitvoer } \ulcorner m\urcorner$
    \end{itemize}

    zijn synoniem met elkaar.
\end{frame}

\subsection{De kracht van berekenbaarheid}
\begin{frame}
    \frametitle{Halting Problem}
\end{frame}

\begin{frame}
    \frametitle{Universaliteits principe}
\end{frame}

\section{De these}
\begin{frame}
    \frametitle{De These}
    \begin{center}
        {\Large
            Every \emph{effectively calculable} function is \emph{computable}
        }
        \\
        Church (1936), Turing (1937)

        Elke \emph{uitrekenbare} functie is \emph{berekenbaar}
    \end{center}
\end{frame}


\section{Voorbij de these}
\subsection{Hypercomputation}
\begin{frame}
    \frametitle{Hypercomputation}

    Oracle machines

    Infinite state

    Transfiniete recursie
\end{frame}
\subsection{Quantum computing}
\begin{frame}
    \frametitle{Quantum computing}

    Church Turing Deutsch

    Wat doet quantum computing
\end{frame}

%\section*{Tragiek in het paradijs}
\begin{frame}
    \begin{center}
        {\LARGE Tragiek in het paradijs}
    \end{center}
\end{frame}


\begin{frame}
    \frametitle{De protagonisten}
    \begin{tabular*}{\textwidth}{c c}
        \includegraphics[width=0.4\textwidth]{Kleene.jpeg} & \includegraphics[width=0.4\textwidth]{example-image-duck} \\
        {\large Stephen Kleene} (1909-1994) & {\large ???} (1897 - 1954)\\
    \end{tabular*}
\end{frame}

\begin{frame}
    \frametitle{De protagonisten}
    \begin{tabular*}{\textwidth}{c c}
        \includegraphics[width=0.4\textwidth]{Kleene.jpeg} & \includegraphics[width=0.4\textwidth]{Post.jpg} \\
        {\large Stephen Kleene} (1909-1994) & {\large Emil Post} (1897 - 1954)\\
    \end{tabular*}
\end{frame}


\end{document}